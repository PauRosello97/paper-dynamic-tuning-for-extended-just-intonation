\documentclass{article}

% Language setting
% Replace `english' with e.g. `spanish' to change the document language
\usepackage[english]{babel}

% Set page size and margins
% Replace `letterpaper' with `a4paper' for UK/EU standard size
\usepackage[letterpaper,top=2cm,bottom=2cm,left=3cm,right=3cm,marginparwidth=1.75cm]{geometry}

% Useful packages
\usepackage{amsmath}
\usepackage{graphicx}
\usepackage[colorlinks=true, allcolors=blue]{hyperref}

\title{Dynamic Tuning for Extended Just Intonation}
\author{Pau Roselló}

\begin{document}
\maketitle

\begin{abstract}
    Your abstract.
\end{abstract}

\section{Introduction}
Kirck defined a two-dimensional pitch-class space \cite{Kirck1987} to map Ben Johnston’s notation \cite{Johnston1977} to just intervals.
Sabat proposed Micromælodeon as a microtuning algorithm using Tenney's Harmonic Distance and a lookup table with 3997 intervals \cite{Sabat2008}.
Stange at al. did a great analysis on previous dynamically adaptive tuning systems which used logical operations and proposed instead a mathematical system of linear equations \cite{Stange2017}.
Trueman et al. studied how a playful interface called bitKlavier could be used to implement adaptive tunings \cite{Trueman2020}.

% \clearpage
\bibliographystyle{ACM-Reference-Format}
\bibliography{references}

\end{document}